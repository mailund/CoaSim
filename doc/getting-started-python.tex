\documentclass{manual}

\title{Getting Started with CoaSim/Python}
\subtitle{An introduction to the simulator CoaSim}
\authors{Thomas Mailund}
\contact{mailund@birc.au.dk}
\company{Bioinformatics ApS}
\toolversion{CoaSim/Python v0.2}

\newcommand{\cM}{\mathrm{cM}}

\begin{document}

\section{About CoaSim}

CoaSim is a tool for simulating the coalescent process with
recombination and geneconversion, under either constant population
size or exponential population growth.  It effectively constructs the
ancestral recombination graph for a given number of chromosomes and
uses this to simulate samples of SNP and micro-satellite haplotypes or
genotypes.

CoaSim comes in two flavours: A graphical user interface version for
easy use by novice users, and a script based version (using either
Guile-Scheme or Python) for efficient batch simulations.  This
document is an introduction to the Python based version.


\subsection{Installing CoaSim}

CoaSim is distributed as RPM files, binary \texttt{tar} files, or as
source code.  For most users, we recommend installing from the RPM
files, since building the tool from source requires setting up the
right build environment and having access to the needed development
tools.  If you are not familiar with UNIX C++ development---using the
Automake suite of tools and Pythons distutils system---we do not
recommend that you try building from source.

\paragraph{Installing the RPM Files.}

The RPM files contains binary versions of the program, compiled to an
Intel x86 Linux platform.  To install the Scheme version, run
\begin{code}
> rpm -Uvh coasim-python-x.y.z-r.i386.rpm
\end{code}
Since the RPM files installs in the directory \verb?/usr/local/?,
installing the RPM package requires root access.

\paragraph{Installing the Binary \texttt{tar} Files.}

The binary \texttt{tar} files contains a pre-compiled version of
CoaSim that should just be un\texttt{tar}ed in the root.  Notice that
you need to have root access to do this:
\begin{code}
> cd /
> tar zxf /path/to/tarfile/coasim-python-x.y.z.linux-i686.tar.gz
\end{code}

If you do not have root access, you can un\texttt{tar} CoaSim in a
different directory, but in that case it is necessary for you to set
your PYTHONPATH so Python can locate the CoaSim module, e.g.\ in bash
\begin{code}
> export PYTHONPATH=/path/to/CoaSim/Python:\$PYTHONPATH
\end{code}
or in csh
\begin{code}
> setenv PYTHONPATH /path/to/CoaSim/Python:\$PYTHONPATH
\end{code}


\paragraph{Installing from the Source Files.}

The source code is distributed in a tar-file; to build the Python version from
the source files untar the file, build the Core module
\begin{code}
> tar zxf coasim-python-x.y.z.tar.gz
> cd coasim-python-version/Core
> ./configure
> make
\end{code}
and then build the Python module:
\begin{code}
> cd ../Python
> python setup.py build
\end{code}

This will build a module that you can import into your Python scripts.
You will need to either install it or make sure you have the path to
it in your PYTHONPATH, e.g.\ in bash
\begin{code}
> export PYTHONPATH=/path/to/CoaSim/Python:\$PYTHONPATH
\end{code}
or in csh
\begin{code}
> setenv PYTHONPATH /path/to/CoaSim/Python:\$PYTHONPATH
\end{code}

\subsection{Running CoaSim}

Once installed, you can load CoaSim into python using the
\texttt{import} command:
\begin{code}
> python
Python 2.3.4 (#1, Feb  2 2005, 12:11:53)
[GCC 3.4.2 20041017 (Red Hat 3.4.2-6.fc3)] on linux2
Type "help", "copyright", "credits" or "license" for more information.
>>> import CoaSim
>>>
\end{code}
to get a short overview of the functionality in this module, use
\begin{code}
>>> help(CoaSim)
\end{code}

\section{Using CoaSim}

Running CoaSim will in most cases consist of three steps: Set up the
simulation parameters, including the markers (type of marker, mutation
rates and similar), demographic parameters, rates for recombination,
etc.; running the simulation obtaining and ARG; and extracting the
needed information from the ARG (\emph{Ancestral Recombination
  Graph}), e.g. the resulting sequences, the timing of the various
events, or the local coalescent trees embedded in the ARG.

Before using the Python based CoaSim \emph{it is necessary that you
  have installed the module}, not just built it.  If the module has
not been properly installed, it will not be able to locate the module
and load it into Python.  If it is not possible to install the modules
globally, you can install the locally but you will then need to set
the \verb?PYTHONPATH? environment variable to let Python know
where the module is installed.

Once CoaSim is successfully installed, it can be loaded into Python
and used as any other python module.  Controlling the simulations
through Python makes CoaSim a very flexible and powerful simulation
tool. However, with flexibility inevitably is associated some
complexity, and while running simple simulations through the Python
interface is quite simple, the more complex simulation tasks require a
bit of knowledge about the Python programming language and the Python
modules supplied with CoaSim.  To keep this `getting started' guide
short, we do not attempt to explain the Python interface to CoaSim in
detail here---for this we refer to the \emph{CoaSim/Python
  Manual}---instead we give a short introduction to running very
simple simulations.

A very simple use of CoaSim is to simulate coalescent trees.  A script
for that is shown here:
\begin{code}
from CoaSim import simulate, SNPMarker

markers = [SNPMarker(0.5,0,1)]
noLeaves = 10

arg = simulate(markers, noLeaves)

tree = arg.intervals[0].tree
print tree
\end{code}
This might look a bit complicated, but if you break it down in the
three phases mentioned at the beginning of this section---setting up
parameters, running a simulation, and analysing the result---it is
really not.

The first line is not really part of any of the three phases, but
simply loads the CoaSim module into Python and imports
\texttt{simulate} and \texttt{SNPMarker} into the global namespace.
The next two lines define the list of markers (polymorphic sites) the
simulated ARG should contain, and the number of leaves the ARG should
contain.  For the markers, we specify a list of a single marker, a
\emph{Single Nucleotide Polymorphism} (SNP) marker, positions at the
middle of the genomic region we consider (position $0.5$), with the
$1$-allele frequency to be between $0$ and $1$, i.e. unrestricted.

This concludes the setup-phase.  The next line is the simulation
phase; it calls \texttt{simulate} to obtain an ARG.  From this ARG we
can, in the third and final phase, extract the list of intervals
sharing the same genealogy (\texttt{arg.intervals})---these are
intervals where no recombination has occurred, in this particular case
there is only one since by default the recombination rate is
$0$---select the first interval, and extract the genealogy for that
interval, which we then print.

As another simple application, consider simulating a list of SNP
sequences.  A script for simulating 100 sequences with 10 SNP markers
in each is shown here:
\begin{code}
from CoaSim import simulate
from CoaSim.randomMarkers import makeRandomSNPMarkers

markers = makeRandomSNPMarkers(10, 0.0, 1.0)
sequences = simulate(markers, 100, rho=400).sequences

print sequences
\end{code}

The first line, once again, loads CoaSim into Python and imports the
\texttt{simulate} function into the global namespace.  The second line
loads another module, \texttt{CoaSim.randomMarkers} and loads the
method \texttt{makeRandomSNPMarkers} from that module into the global
namespace.  This method is used to create a list of
randomly\footnote{Randomly here means uniformly distributed on the
  interval being simulated.} positioned SNP markers in the next line.
Here we choose a list with $10$ markers, with the $1$-allele frequency
once again between $0$ and $1$.  The next line simulates the ARG, this
time with $100$ leaves, and then extracts the sequences from the ARG.

The keyword argument, \texttt{rho}, sets the scaled recombination rate
$\rho=4N_er$ to 400 (which for an effective population size $N_e$ of
$10,000$ roughly correspond to $1\,\cM$).

Printing sequences with the \texttt{print} command will print a list
of lists; this is a format that is very simple for Python to read and
manipulate, but is usually not suitable for other analysis tools.  To
print the sequences in a more traditional form of a sequence per line,
with the sequences printed as space-separated numbers, we can use the
\texttt{CoaSim.IO} module like this:
\begin{code}
from CoaSim.IO import printMarkerPositions, printSequences
printMarkerPositions(markers)
printSequences(sequences)
\end{code}
Here, the positions of the marker list are first output as a line of
space-separated numbers, and then the sequences are output.  In both
cases, the output is to the terminal, but using a second argument to
the two print functions, we can direct the output to files.  E.g.\ to
write the positions to \texttt{positions.txt} and the sequences to
\texttt{sequences.txt}, we use:
\begin{code}
from CoaSim.IO import printMarkerPositions, printSequences
printMarkerPositions(markers, open('positions.txt','w'))
printSequences(sequences, open('sequences.txt','w'))
\end{code}

A common setting is simulating a set of sequences and then split them
into cases and controls based on a trait mutation.  A trait mutation
is really just any mutation, typically on a bi-allelic marker.  As
such, a disease locus can be modelled simply as a SNP marker.  SNP
markers, however, model a certain ascertainment bias in the
simulations---one that matches the typical SNP selection bias---when
the mutant allele frequency is restricted.\footnote{When the mutant
  allele frequency is unrestricted, a trait marker and a SNP marker
  has exactly the same semantic.  Due to the different
  rejection-sampling approaches to restricting the allele frequencies
  for SNP and trait markers, respectively, restricting trait markers
  can be significantly more CPU demanding than restricting SNP
  markers.}  To simulate without this bias, we can use a
\texttt{TraitMarker}.

A \texttt{TraitMarker} instance is created just as a
\texttt{SNPMarker} instance, and takes the same arguments---position
and range of the mutant allele frequency---and can be added to the
marker list for a simulation just as a SNP marker.

To simulate sequence data with a single disease affecting marker, we
can use:
\begin{code}
from CoaSim import simulate, TraitMarker, insertSorted
from CoaSim.randomMarkers import randomPosition, makeRandomSNPMarkers
from CoaSim.diseaseModelling import singleMarkerDisease, split

markers = makeRandomSNPMarkers(10, 0.0, 1.0)
traitMarker = TraitMarker(randomPosition(), 0.2, 0.4)
markers,traitIdx = insertSorted(markers, traitMarker)

sequences = simulate(markers, 100, rho=400).sequences
diseaseModel = \(\backslash\)
    singleMarkerDisease(traitIdx, wildTypeRisk=0.1, mutantRisk=0.3)
affected, unaffected = split(diseaseModel,sequences)
\end{code}

The first three lines just load modules and import functions and
classes into the global namespace, as before.  We then create $10$
randomly positioned SNP markers, as we did earlier, but now also a
single randomly positioned trait marker.  We insert this into the list
of markers, to get a new list (of $11$ markers, $10$ SNP markers and
the trait marker) and the index where the trait marker can be found.
We then simulate $100$ sequences and split them into affected and
unaffected using the \texttt{split} function from module
\texttt{CoaSim.diseaseModelling}.

The first argument to \texttt{split} is a disease model.  This
parameter determines how the sequences should be interpreted and how
disease status is determined.  In this case we use a simple
single-locus model, where the alleles of the trait marker determines
the disease status based on the probability of a wild-type being
affected (the \texttt{wildTypeRisk}) and the probability of a mutant
being affected (the \texttt{mutantRisk}).

This disease model considers the sequences haploid, but other models
consider the sequences pairwise\footnote{The sequences are considered
  pairwise in the order they appear in the sequence list, i.e.\ for
  $i=0,2,4,\ldots$, \texttt{sequence[$2i$]} and
  \texttt{sequence[$2i+1$]} are paired.} as diploid individuals and
determines disease status from this.  For example, to simulate a
dominant or recessive diploid disease, we can use
\begin{code}
from CoaSim.diseaseModelling import dominantMode, split
diseaseModel = dominantModel(traitIdx)
\end{code}
or
\begin{code}
from CoaSim.diseaseModelling import recessiveMode, split
diseaseModel = recessiveModel(traitIdx)
\end{code}
respectively.



\section{Contact}
\label{sec:contact}

For any comments or questions regarding CoaSim, please contact Thomas
Mailund, at \href{mailto:mailund@mailund.dk}{mailund@mailund.dk} or
\href{mailund@birc.au.dk}{mailund@birc.au.dk}.


\end{document}

% LocalWords:   CoaSim geneconversion haplotype haplotypes ARG
